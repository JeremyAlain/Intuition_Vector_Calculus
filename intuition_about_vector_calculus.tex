\documentclass[12pt,a4paper]{article}
\usepackage[latin1]{inputenc}
\usepackage{amsmath}
\usepackage{amsfonts}
\usepackage{amssymb}
\usepackage{graphicx}
\author{Jeremy Scheurer}
\title{Intuition about Vector Calculus}

% The content provided in this document is all taken from the website mathinsight.org and was solely summarized for private usage.

\begin{document}
	\maketitle
	\tableofcontents
	\let\thefootnote\relax\footnote{The content is mainly a summary of different websites(mainly http://mathinsight.org/ and http://betterexplained.com). Sometimes sentences are cited one-to-one, but I have refrained from making citations and a bibliography because this paper's goal is not to be published but to use privately.}
	\newpage
	
	\section{Flux}
		\begin{itemize}
		\item Flux is the amount of "something"(a force, an electric field, potatoes...) passing through a surface. 
		\item The \textbf{total flux} depends on:
			\begin{itemize}	
				\item \textbf{Strength of the field}: The source of the flux has a huge impact on the total flux. Doubling it's strenght will double the amount of potatoes you shoot through your surface.
				\item \textbf{Size of the surface it passes through}: The bigger the surface, the more fits through.
				\item \textbf{Their orientation}: When the surface faces the source(they are orthogonal), all of the potatoes can pass the surface. As the surface titls away from the field, less potatoes can enter the surface an the flux decreases. 
			\end{itemize}
		
		\item \textbf{Vector Field}: This is the source of the flux that exerts some force on the surface. 
		
		\item \textbf{Surface}: This is the boundary the flus passes through. It can be a sphere, a plane or the top of a bucket. Notice that the top of a bucket is empty, but that we can measure the flux passing through this empty region.
		
		\item \textbf{Timing}: We measure the flux at a specific time. Because the flux can change, we have to freeze time and look at it at this moment. 
		
		\item We say that we have \textit{positive flux} when it leaves the closed surface and \textit{negative flux} when it enters a closed surface.
		
		\item Now let's get to the math and derive a formula. We want to know how much of a vector field is passing through our surface, taking the magnitude, orientation, and size into account. \\ \\
		\textit{Total flux = Field Strength $\cdot$ Surface Size $\cdot$ Orientation}\\ \\
		Because the vector field is not the same everywhere we will look at tiny little surface pieces a time and add them up.\\ \\
		\textit{Total flux = (Field Strength $\cdot$ dS $\cdot$ Orientation) for every dS) = \\ Integral(Field Strength $\cdot$ Orientation $\cdot$ dS)} \\  \\
		
		\includegraphics[width=0.8\textwidth, height = 150px]{flux.png}\\
		\textit{We see that the flux is parallel with the top of the surface. But the left surface gets maximal flux as it is directed orthogonal to it.}

		\item Mathematically we denote a surface with its normal vector which sticks out of the surface orthogonally. This means the normal vector of the left surface is parallel to the flux.
		\item As we are trying to measure how much the strenght of the vector field passes through the surface at different angles, you could rephrase it as how does the angle between the source and the surface affect the strenght of the flux on the surface. How could we mathematically denote that?\\
		Well that's easy, we use the dot product. It gives us a number between 0 and 1 that tells us what percentage of the field is passing through the surface.
		\item If we take together both aspects we get following new formula.
		Total flux = Integral(Vector Field Strength dot normal Vector)dS \\
		$= \int_{S}^{} \overrightarrow{F} \cdot \overrightarrow{n} dS$
		
		
		\end{itemize}
	
	\newpage
	
	\section{Divergence}
	
		\begin{itemize}
			\item Divergence is nothing else than "flux density" meaning the net flux per unit volume. It tells you how much flux is entering or leaving a specific point. You can also think about it as flux expansion(positive divergence) or flux contraction(negative divergence).
			
			\item Imagine you could talk to a specific point. If he saw flux entering he would tell you that ther is negative divergence. If he saw flux leaving he would say there is positive divergence. Also the bigger the flux densitivy(positive or negative), the stronger the flux source or sink. If the divergence is zero, it menas that there is no net flux change. 
			
			\item Mathematically speaking: \\
			\textit{Divergence}$ = \frac{Flux}{Volume}$ \\ \\ 
			So if you imagine that you had a tiny little cube(in the space $R^3$) you'd have to calculate the flux change in all directions to get divergence: \\ \\
			\textit{Total flux change = (field change in X direction) + (field change in Y direction) + (field change in Z direction)} \\
			$ \leftrightarrow Divergence = \lim\limits_{Vol \rightarrow 0} \frac{Flux}{Vol}$ \\
			$\leftrightarrow Divergence = \bigtriangledown (Gradient) =  \frac{\partial F_1}{\partial x} + \frac{\partial F_2}{\partial y} + \frac{\partial F_3}{\partial z}$\\
			
			\includegraphics[width=0.7\textwidth, height = 200px]{divergence.png}\\
			\textit{Divergent vector field with embedded sphere in three-dimensional space. The vector field is expanding thus the divergence is positive.}
			
		\end{itemize}
	
	\newpage
	\section{Line Integral of scalar-valued functions}
	
	\begin{itemize}
	
	\item The idea is similar to the one of an Integral over a two dimensioal Surface. As analogy you can imagine to calculate the mass of a wire from its density. 
	
	\item Assume you have a wire whose density is not constant over its length. We have a function $C(t)$ which describes a certain point on the wire. How can you calcuate it's mass from its density?
	
	\item We can segment the wire into lots of smaller segments and calculate their mass according to the specific densities. The length of the i-th segment is defined as: $||c(t_i) - c(t_i -1)||$. \\
	The densitiy for a specific point is defined as: $f(c(t_i))$
	
	\item The mass of a segment is just the Linesegment $\cdot$ its density: \\
	$f(c(t_i))\cdot||c(t_i) - c(t_i -1)||$ \\
	To calculate the total mass of the whole wire you just have to sum up all the masses of the segments: $\sum_{i}^{n} f(c(t_i))\cdot||c(t_i) - c(t_i -1)||$
	
	\item To turn this into an integral we define: $ \Delta t_i = t_i - t_{i-1}$. Now multiply and divide each term by $\Delta t_i$ and obtain the more complicated looking expression: \\
	$\sum_{i}^{n} f(c(t_i))\cdot ||c(t_i)-c(t_{i-1})|| = \sum_{i}^{n}     ||\frac{c(t_{i-1} + \Delta t_i)-c(t_{i-1})}{\Delta t_i}|| \cdot \Delta t_{i-1}$\\
	If you are confused by the counter just realize, that $c(t_{i-1} + \Delta t_i)-c(t_{i-1} = c(t_i) - c(t_i -1)$
	
	\item You realize, that the term in the expression $||.||$ is actually the definition of the derivate off $c(t)$. The counter is nothing else than a function c(t) applied to an interval and the denominator is that exact interval. So if you apply it to a two-dimensional function it's nothing else than $\frac{\Delta y}{\Delta x}$ which defines the secant. 
	
	For comparison here is the definition of the definition of a Secant: \\
	$\frac{f(x_0 + \Delta x) - f(x_0)}{\Delta x}$


	\item If we let $\Delta t_i \rightarrow 0$ and $n \rightarrow \infty $ the above Riemann sum converges to the integral:
	$\int_{a}^{b} f(c(t))\cdot ||c'(t)|| dt$ which is often denoted as $\int_{c}^{}f ds$.
	\end{itemize}
	
	\section{Introduction to a line integral of a vector field}
	
	\begin{itemize}
	\item The only change to the chapter above is that now we want to integrate a vector valued curve function along a curve. They are usually represented by a vector field.\\	\includegraphics[width=0.8\textwidth, height = 200px]{Field_Helix.png}
	\\ $\textit{A very easy interpretation is to imagine the amount of work that a}$\\ $\textit{force field does on a particle as it moves along a curve.}$
	
	\item Imagine you have a Bead that is fixed to move along a helix defined as $c(t)$. The green rectangle is a magnet inducing a magnetic field $F(x,y,z)$(illustrated by the green arrows). The line integral of a vector field can be thought of, as work on the bead. 
	
	\item When the bead changes its position along the function, the force exorted by the magnetic field, and thus also the work, changes.\\
	
	Work is defined as $F\cdot s$ (force in direction of movement). For example, when the movement is 90 degrees from the direction of the force, the magnetic field does no work at all. So how can we get the direction of the bead? \\
	
	We just use its direction given by the velocity $c'(t)$. We denote the unit vector in the direction of the movement as $T(t) = \frac{c'(t)||}{c'(t)||}$. If you think of it, this absolutely makes sense as the derivate of $c(t)$ defines teh tangent vector to the path. 
	
	\item  The component of force in direction of the movement is simply the force of the magnetic field at a specific point on the function $c(t)$ along the tangent of $c(t)$: $F(c(t))\cdot T(t)$.(for future reference we will use$F\cdot T)$ \\
	
	Per definition we know that if we take the product of force and distance we get work. So $F\cdot T$ actually denotes the work per unit length along the helix. \\
	To get the total work we need to ake the line integral of this scalar-valued function. \\
	Work = $\int_{C}^{} F\cdot T ds$ (where C is the path along the function$c(t)$)
	
	\item To derive a final formula we go back to the first chapter where we looked at scalar-valued line integrals and take the end-formula that we derived.\\
	$\int_{C}^{} f ds = \int_{a}^{b} f(c(t))\cdot ||c'(t)|| dt$\\
	If we replace f with our new equation for work $F\cdot T$ we get:\\
	
	$\int_{C}^{} F\cdot T ds = \int_{a}^{b}F(c(t))\cdot T(t)\cdot ||c'(t)|| dt$\\
	
	We remember that we defined $T(t) =  \frac{c'(t)}{||c'(t)}$. If we replace $T(t)$ we get: \\
	
	$\int_{a}^{b}F\cdot T ds = \int_{a}^{b} F(c(t))\cdot\frac{c'(t)}{||c(t)||}\cdot ||c'(t)|| dt \\= \int_{a}^{b} = F(c(t))\cdot c'(t) dt = \int_{C}^{}F ds$.\\
	(because we usually write $T ds$ as ds.)
	\end{itemize}
	\newpage
	
	\section{Line integrals as circulation (Macroscopic)}
	\begin{itemize}
	
	\item In the previous chapter we learned how a vector field F over an oriented curve C "adds up" the component of the vector field that is tangential to the curve. In simple words, this defines how much the vector field is aligned with the curve. But what do we do if the curve is closed?
	
	\item The line inegral then indicates how much the vector fields tends to circulate around the curve C.\\
	$\int_{C}^{}F\cdot ds$ = $\oint_{C}^{}F\cdot ds$ (other notation) = circulation of F around C 
	\enlargethispage{\baselineskip}
	\item The circulation can be positive or negative according to the "circling"-direction of the vector field around the curve C. It is positive if it circles with the Curve and negative if it circles counter-clockwise to the Curve.
	\item You can also imagine this formula in a more intuitive way. Circulation is the integral of a vector field along a path. This means you add up how much the field pushes you along a path. 
	If you have a function $F(t)$ that defines the poisition you can take the time derivateive to ge the velocity at that poistion. Why do you want the velocity vector? Because you do not want to know where you are but where you are going and how much the vector field pushes you in this direction(in a river are you going upstream or downstrem?). So if F(s) is the position, dS is the velocity. We get the same formula as above:\\
	Total pushing force = Circulation = $\int F(r) \cdot dr.$	\\ $\newline$
	\includegraphics[width=0.8\textwidth, height = 200px]{circulation.png}	
	
	$\textit{In this example the circulation of the vector field circulates clockwise in}$\\ $\textit{the opposite direction of the Curve.}$ 
	\item You can also imagine the cirulation by measuring the total "push" you get when going along a path such as a circle. Or if you had a paper boat in a whirlpool, the circulation would be the amount of force that pushed it along as it went in a circle. 
	\end{itemize}
	
	\newpage
	
	\section{The curl (Microscopic)}
	\begin{itemize}
		
		\item The curl is simply the circulation per unit are, circulation density, or rate of rotation at a signe point. Imagine you would shrink the whirlpool(from the Circulation chapter) to a tiny size but keeping the force the same. You will have a a lot of power in a small spot $\rightarrow$ large curl. If you'd widen the whirlpool and keep the force the same you will have the force distributed over a larger area. $\rightarrow$ small curl. 
		
		\item Imagine that in the chapter above you had a small ball circling around a curve, driven by a vector field. Now the small ball is at a spot where there is no circulation around the vector field. But because the vector field still exerts force on the ball, it will circly on the spot. 
		
		\includegraphics[width=0.6\textwidth, height = 200px]{curl.png}\\
		\textit{The green arrow is the curl of the vector field. For explaination see below.}
		
		\item Suppose we have a current of water and we want to determine if there is a curl or not(so is there microscopically any twisting or pushing force?). To test that we put a paddle wheel into the water and notice if it turns. 
		
		\includegraphics[width=0.7\textwidth, height = 200px]{paddle_wheel.png}\\
		\textit{If the paddle does turn the vector field has a curl at
		 that point. If it doesn't turn then there isn't a curl.}
		
		\item The paddle wheel will start to turn if there is more force on one side than on the other. Thus if the vector field is uneven or assymetric, at that specific point, there exists a curl. 
		\item Because there is this uneven, unexpected twisting enforced by the vector field we call this field \textbf{not conservative}. A \textbf{conservative} field would be a field with zero curl. It also has the property that its line integral is path independent, meaning that choosing any two points for the line integral won't change the result. A conervative field is also said to be fair. If you'd float down a river, you could get a free ride down but then you would have to put in work to get up the stream again. In a non-conservative field you can really get a free ride by going with the "random" current of the water. 
		\item The curl is not a scalar like the divergence, but a vector. This means that it has a magnitude and a direction. The magnitude is simply the force of the twist. The direction is defined as the axis, around which the paddle wheel turns. So if you think about it, there could be two directions(if you turned the paddle wheel upside down). The curl's direction is defined to rotate counterclockwise, which gives you then its direction(you can test that with the right hand rule).
		
		\item Mathematically speaking we can define the curl as circulation per area with its vector pointing out of the page(if the curl is counterclockwise). We obtain following formula:\\
		Curl = $\frac{circulation}{area} = \frac{\int F(r \cdot dr)}{\int S}$
		
		\subsection{Components of the curl(another approach)}
		
		\item We go back to the first situation described above, where the ball is in the middle of a fluid with a flow given by F. It experiences a curl but no circulation. Using the right hand rule we can denote the vector curl F and the magnitude is defined as $||curl F ||$.
		
		\item We find ourself in three dimensional space. This means that the curl F has an x, y and z component. If we let v = curl F, then we could write curl F in terms of components as:\\
		$curl F = v = V_1i + v_2j + v_3$
		
		\item \textbf{Z-component of curl}\\
		Here we fixate the z-axis on a rod and thus allow the ball only to rotate around z-axis(so its only influenced by the x and y component of F). This rotation corresponds to $v_3k$, where $v_3$ is the component of the curl in the z direction. The speed of the rotation $ = |v_3|.$ 
		\includegraphics[width=0.6\textwidth, height = 200px]{z-component.png}\\
		\textit{The ball is rotating in a counterclockwise direction. Using the right hand rule we see that the curl is positive}\\
		
		The curl is not parallel to the z-axis, this implies that the magnitude of the z-component v3 is smaller than the magnitude of curl F(the ball rotates slower than it did when it was allowed to rotate freely). \\
		What properties of vector field F will cause the sphere to spin even now that it cannot move freely? The rotation is solely dependent on the x-component $F_1$ and the y-component $F_2$. \\
		\includegraphics[width=0.6\textwidth, height = 200px]{xy-plane.png}\\
		\textit{This image illustrates the xy-plane with the z-axis coming right out of the paper.}\\
		
		You can easily imagine what you need to do to create a counterclockwise rotation.The vector field component $F_1$(bottom) needs to have a stronger push than $F_1$(top). Similarly $F_2$(right) needs to have a stronger push than $F_2$(left). The curl is supposed to correspond to "microscopic circulation," so we view the sphere as being very small. The increase of $F_2$ as we move a tiny bit to the right(and similarly the \textbf{decrease} of $F_1$ as we move to the top)  is captured by the partial derivative of $F_2$ with respe cto to x(of $F_1$ in respect to y). It turns out that the effects of these changes in F simply add up. We can catch that in the following formulas.\\
		\begin{center}
		$v_3 =$ curl$F \cdot k = \frac{\partial F_2}{\partial x} - \frac{\partial F_1}{\partial y}$
		\end{center}
		\footnote{If you wonder why it's a substraction and not an addition think again. We said that $F_1$ has to be stronger in the bottom. Since this corresponds to $F_1$ decreasing as y increases, we need a negative $\frac{\partial F_1}{\partial y}$}.
		\newpage
		\item If you replay that game and keep the righthand-rule, but rename the axis(in the same order but with the new fixated axis pointing out of the paper), you can get the formulas for the fixated y-axis and x-axis. \\
		
		\begin{center}
			y-axis: $v_2$ = curl $F \cdot j = \frac{\partial F_1}{\partial z} - \frac{\partial F_3}{\partial x}$\\ $\newline$
			x-axis: $v_3$ = curl $F \cdot i = \frac{\partial F_3}{\partial y} - \frac{\partial F_2}{\partial z}$\\ $\newline$
			
			$ \leftrightarrow$ curl F = ($\frac{\partial F_2}{\partial x} - \frac{\partial F_1}{\partial y}$,$\frac{\partial F_1}{\partial z} - \frac{\partial F_3}{\partial x}$,$\frac{\partial F_3}{\partial y} - \frac{\partial F_2}{\partial z}$)
		\end{center}
		
		\newpage
		
		
		
		
		
		
		


		
		 
	\end{itemize}
	
	\newpage
	
	\section{Gradient Theorem}
	
	\newpage
	
	\section{Green's Theorem}
	
	\newpage
	
	\section{Stoke's Theorem}
	
	\newpage
	
	\section{Gauss Theorem(Divergence Theorem)}
	
	
	

	
	
	

	

\end{document}